%======================================================================
%----------------------------------------------------------------------
%sergiobuj's ieee format template
%======================================================================
\documentclass[%
	%draft,
	%submission,
	%compressed,
	final,
	%
	%technote,
	%internal,
	%submitted,
	%inpress,
	%reprint,
	%
	%titlepage,
	notitlepage,
	%anonymous,
	narroweqnarray,
	inline,
	twoside,
         %invited,
	]{ieee}

\newcommand{\latexiie}{\LaTeX2{\Large$_\varepsilon$}}

%Español
	\usepackage[utf8]{inputenc}
	\usepackage[spanish]{babel}
%loñapsE

%\usepackage{ieeetsp}	% if you want the "trans. sig. pro." style
%\usepackage{ieeetc}	% if you want the "trans. comp." style
%\usepackage{ieeeimtc}	% if you want the IMTC conference style

% Use the `endfloat' package to move figures and tables to the end
% of the paper. Useful for `submission' mode.
%\usepackage {endfloat}

% Use the `times' package to use Helvetica and Times-Roman fonts
% instead of the standard Computer Modern fonts. Useful for the 
% IEEE Computer Society transactions.
%\usepackage{times}
% (Note: If you have the commercial package `mathtime,' (from 
% y&y (http://www.yandy.com), it is much better, but the `times' 
% package works too). So, if you have it...
%\usepackage {mathtime}

% for any plug-in code... insert it here. For example, the CDC style...
%\usepackage{ieeecdc}

\begin{document}

%----------------------------------------------------------------------
% Title Information, Abstract and Keywords
%----------------------------------------------------------------------
\title[Avance práctica 2]{%
       Avance práctica 2 \\  Organización de Computadores}

% format author this way for journal articles.
% MAKE SURE THERE ARE NO SPACES BEFORE A \member OR \authorinfo
% COMMAND (this also means `don't break the line before these
% commands).
\author[]{Sebastián Arcila Valenzuela (\textit{sarcilav@eafit.edu.co})
\and{}y Sergio Botero Uribe (\textit{sbotero2@eafit.edu.co}).
}

% format author this way for conference proceedings
%\author[PLETT AND KOLL\'{A}R]{%
        %Gregory L. Plett\member{Student Member},\authorinfo{%
        %Department of Electrical Engineering,\\ 
        %Stanford University, Stanford, CA 94305-9510.\\
        %Phone: $+$1\,650\,723-4769, email: glp@simoon.stanford.edu}%
%\and{}and%
%\and{}Istv\'{a}n Koll\'{a}r\member{Fellow}\authorinfo{%
        %Department of Measurement and Instrument Engineering,\\ 
        %Technical University of Budapest, 1521 Budapest, Hungary.\\
        %Phone: $+$\,36\,1\,463-1774, fax: +\,36\,1\,463-4112, 
        %email: kollar@mmt.bme.hu}
%}

\journal{ST0254-031 Organización de Computadores}
\titletext{, \today} 
%\ieeecopyright{0018--9456/97\$10.00 \copyright\ 1997 IEEE}
%\lognumber{xxxxxxx}
%\pubitemident{S 0018--9456(97)09426--6}
%\loginfo{Manuscript received September 27, 1997.}
%\firstpage{0}

%\confplacedate{Ottawa, Canada, May 19--21, 1997}

\maketitle               

\begin{abstract} 
Avance número 1 de la práctica 2, sobre paralelismo.
\end{abstract}

\begin{keywords}
Práctica 2, organización de computadores, paralelismo, Logisim y procesador.
\end{keywords}

%----------------------------------------------------------------------
% SECTION I: Introduction
%----------------------------------------------------------------------
\section{Comentarios}

\PARstart Para este primer avance de la prácitca todavía no se tiene ninguna implementación para empezar
a discutir sino más bien una serie de ideas que se espera se vayan desarrollando más fuertemente en el
transcurso de la semana.\\
Lo primero que se debe pensar para la práctica de paralelismo es que el objetivo queda en condicionar dos
o más procesadores, aunque sería interesante revisar la idea de obtener el mismo desempeño simplemente
modificando los componentes que necesitamos que hagan más trabajo, en este caso la unidad aritmética lógica,
sin perder el horizonte de los procesadores.

Una buena forma de mirar el objetivo de la práctica, sería comparando un vector de números enteros con una
hoja de papel, y la idea es que se vaya doblando por la mitad hasta llegar al punto en que no se puede doblar más,
que para el caso del vector sería la suma. Por estar doblando el vector a la mitad queremos decir que debe haber una
unidad de procesamiento que se encargue de la mitad de ese vector y así sucesivamente hasta que al final una de ellas
queda con el resultado final. Es un ejemplo sencillo para exponer la forma en que entendimos la práctica y
como hasta este punto nos gustaría que funcionara.
 
 Para el manejo del procesador, la aproximación más acertada sería poder hacerlo con microprogramación, en principio
 con las instrucciones básicas y esperar poder ampliar el conjunto en la medida de lo posible.

%----------------------------------------------------------------------
% SECTION II: Estructura de la práctica
%----------------------------------------------------------------------


%----------------------------------------------------------------------
% SECTION III: Dificultades
%----------------------------------------------------------------------


%----------------------------------------------------------------------
\section{Tareas por realizar}

Lista de las actividades que faltan para terminar la práctica y
que además serían interesantes realizar:
\begin{itemize}
\item Revisar métodos eficientes para realizar sumas de vectores.
\item Investigar soluciones a nivel de algoritmos que ofrezcan buen desempeño.
\item Procesador.
\item Buscar métodos aplicados a vectores que se puedan mejorar mediante paralelismo (con al intención de ampliar la
funcionalidad, pero no se promete nada).
\end{itemize}

%----------------------------------------------------


%\begin{thebibliography}{1}

%\end{thebibliography}

%----------------------------------------------------------------------

\end{document}
