\section{Detalles de diseño de construcción}

Para esta práctica se tuvieron varias ideas principales, algunas de ellas se basaban en un procesador completamente
diferente, otras tenían como idea principal el uso de un BUS más amplio, otras sólo pensaban sobrecargar una unidad
aritmetico-lógica mientras que otras mucho más arriesgadas trataban de incorporar algo de todas.\\
Algunas aproximaciones fueron:
\begin{itemize}

\item La primera aproximación fue la de crear una ALU que tuviera cuatro entradas y dos salidas de registro, esto llevaría
a usar unos 8 registros para poder tener algunos para propósito general y evitar el hecho de que todo el procesador se
viera involucrado en la operación de sumatoria de los valores del vector. Esta idea empezó a demostrar que era bastante
complicada ya que el número de señales de la unidad de control se incrementaba para el manejo de las entradas y salidas
de los registros. Hasta este momento no habíamos encontrado un error que después haría que se cambiara todo el
procesador.

\item Para atacar el problema de las señales de la unidad de control, se pensó en sobrecargar una ALU, hacer que la operación
de suma de vector se manejara completamente allí, que se tuvieran un conjunto de registros exclusivos para esa operación y que
las señales fueran totalmente independientes de la microprogramación. El error ahora sería que las señales de la unidad de control
se interrumpirían con las que crearía la ALU. Hasta aquí todavía nos estabamos centrando en la ALU como el componente que
haría la diferencia en el paralelismo.

\item Del resto de las ideas y los diseños, la que parecía ser la que daría mejores resultados fue la de implementar una
ALU con ciertos registros independiente de la normal que además sería controlada por una unidad de control independiente.
Esta unidad de control debía ser activada por la unidad de control del procesador  y que luego esta le pasara el `control'
nuevamente a la unidad del procesador.
A este punto ya nos habíamos dado cuenta que los diseños anteriores no estaban atacando el problema para lograr
paralelismo sino que sólo se estaban haciendo muchas sumas en un Tick de reloj, pero nada de paralelismo.

\end{itemize}