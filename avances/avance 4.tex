%======================================================================
%----------------------------------------------------------------------
%sergiobuj's ieee format template
%======================================================================
\documentclass[%
	%draft,
	%submission,
	%compressed,
	final,
	%
	%technote,
	%internal,
	%submitted,
	%inpress,
	%reprint,
	%
	%titlepage,
	notitlepage,
	%anonymous,
	narroweqnarray,
	inline,
	twoside,
         %invited,
	]{ieee}

\newcommand{\latexiie}{\LaTeX2{\Large$_\varepsilon$}}

	

%Español
	\usepackage[utf8]{inputenc}
	\usepackage[spanish]{babel}
%loñapsE

%\usepackage{ieeetsp}	% if you want the "trans. sig. pro." style
%\usepackage{ieeetc}	% if you want the "trans. comp." style
%\usepackage{ieeeimtc}	% if you want the IMTC conference style

% Use the `endfloat' package to move figures and tables to the end
% of the paper. Useful for `submission' mode.
%\usepackage {endfloat}

% Use the `times' package to use Helvetica and Times-Roman fonts
% instead of the standard Computer Modern fonts. Useful for the 
% IEEE Computer Society transactions.
%\usepackage{times}
% (Note: If you have the commercial package `mathtime,' (from 
% y&y (http://www.yandy.com), it is much better, but the `times' 
% package works too). So, if you have it...
%\usepackage {mathtime}

% for any plug-in code... insert it here. For example, the CDC style...
%\usepackage{ieeecdc}

\begin{document}

%----------------------------------------------------------------------
% Title Information, Abstract and Keywords
%----------------------------------------------------------------------
\title[Avance  práctica 2]{%
       Avance 4 práctica 2 \\  Organización de Computadores}

% format author this way for journal articles.
% MAKE SURE THERE ARE NO SPACES BEFORE A \member OR \authorinfo
% COMMAND (this also means `don't break the line before these
% commands).
\author[]{Sebastián Arcila Valenzuela (\textit{sarcilav@eafit.edu.co})
\and{}y Sergio Botero Uribe (\textit{sbotero2@eafit.edu.co}).
}

% format author this way for conference proceedings
%\author[PLETT AND KOLL\'{A}R]{%
        %Gregory L. Plett\member{Student Member},\authorinfo{%
        %Department of Electrical Engineering,\\ 
        %Stanford University, Stanford, CA 94305-9510.\\
        %Phone: $+$1\,650\,723-4769, email: glp@simoon.stanford.edu}%
%\and{}and%
%\and{}Istv\'{a}n Koll\'{a}r\member{Fellow}\authorinfo{%
        %Department of Measurement and Instrument Engineering,\\ 
        %Technical University of Budapest, 1521 Budapest, Hungary.\\
        %Phone: $+$\,36\,1\,463-1774, fax: +\,36\,1\,463-4112, 
        %email: kollar@mmt.bme.hu}
%}

\journal{ST0254-031 Organización de Computadores}
\titletext{, \today} 
%\ieeecopyright{0018--9456/97\$10.00 \copyright\ 1997 IEEE}
%\lognumber{xxxxxxx}
%\pubitemident{S 0018--9456(97)09426--6}
%\loginfo{Manuscript received September 27, 1997.}
%\firstpage{0}

%\confplacedate{Ottawa, Canada, May 19--21, 1997}

\maketitle               

\begin{abstract} 
Avance número 4 de la práctica 2, sobre paralelismo.
\end{abstract}

\begin{keywords}
Práctica 2, organización de computadores, paralelismo, Logisim y procesador, ALU, sumas parciales.
\end{keywords}

\section{Avances}

Para esta entrega parcial de la práctica se llegó a una conclusión para trabajar los formatos de instrucciones.
El procesador va a trabajar con una memoria RAM de $128 bytes$ lo que nos lleva a trabajar con un tamaño
dirección de $7 bits$.\\

Por ahora sólo se llevan implementadas las operaciones fundamentales.

La práctica se está llevando en el repositorio git en: http://github.com/sergiobuj/paralelismo .\\ \\



\begin{itemize}



\item Formato de instrucciones.


\begin{table}[h]
%\caption{MovM}
\begin{center}
\title{Tamaño de instrucciones: 14 bits.}
\begin{tabular}{|c|c|c|c|c|c|c|c|c|c|c|c|c|c|}
\hline
13& 12& 11& 10& 9& 8& 7& 6& 5& 4& 3& 2& 1& 0\\
\hline
\end{tabular}
\end{center}
\end{table}%

Los bits 13-11 serán para el código de operación en todas las instrucciones.\\
\subitem $\triangleright$ Operaciones de transferencia.

\begin{table}[h]
%\caption{MovM}
\begin{center}
\title{MovM}
\begin{tabular}{|c|c|c|c|c|c|c|c|c|c|c|c|c|c|}
\hline
13& 12& 11& 10& 9& 8& 7& 6& 5& 4& 3& 2& 1& 0\\
\hline
\end{tabular}
\end{center}
\end{table}%
Los bits 10-8 serán usados para indicar el registro que estará involucrado en la
transferencia, sea desde hacia memoria.\\
El bit 7 indicará la dirección de esta transferencia, 1) desde registro a memoria y 0)
de memoria a registro.\\
Los bits 6-0 serán para la dirección de memoria.\\


\begin{table}[h]
%\caption{MovM}
\begin{center}
\title{MovR}
\begin{tabular}{|c|c|c|c|c|c|c|c|c|c|c|c|c|c|}
\hline
13& 12& 11& 10& 9& 8& 7& 6& 5& x& x& x& x& x\\
\hline
\end{tabular}
\end{center}
\end{table}%
Los bits 10-8 serán para el registro de origen y los bits 7-5 serán para el registro
de destino, muy similar al formato de las instrucciones de AT\&T.\\


%\end{itemize}



\subitem $\triangleright$ Operaciones de aritméticas.

\begin{table}[h]
%\caption{MovM}
\begin{center}
\title{suma}
\begin{tabular}{|c|c|c|c|c|c|c|c|c|c|c|c|c|c|}
\hline
13& 12& 11& 10& 9& 8& 7& 6& 5& x& x& x& x& x\\
\hline
\end{tabular}
\end{center}
\end{table}%
Los bits 10-8 serán para el registro de origen y los bits 7-5 serán para el registro
de destino, donde quedará almacenada la suma de los dos registros.\\


\begin{table}[htpd]
%\caption{MovM}
\begin{center}
\title{sumaV}
\begin{tabular}{|c|c|c|c|c|c|c|c|c|c|c|c|c|c|}
\hline
13& 12& 11& 10& 9& 8& 7& 6& 5& 4& 3& 2& 1& x\\
\hline
\end{tabular}
\end{center}
\end{table}%
Los bits 10-8 serán para el registro donde se indica el tamaño del vector.\\
Los bits 7-1 indican la posición en memoria del primer elemento del vector.
Actúa de igual forma que un apuntador.


\end{itemize}

\section{Tareas por realizar}
Lista de las actividades que faltan para terminar la práctica y
que además serían interesantes realizar:
\begin{itemize}
\item Montar el juego de señales
\item Procesador.
\end{itemize}

\end{document}
